\documentclass[12pt, letterpaper]{article}
\usepackage[top=4cm, left=4cm, right=3cm, bottom=3cm]{geometry}
\usepackage{graphicx}
\linespread{1.25}
\graphicspath{ {./} }

\begin{document}
\begin{titlepage} 
  \begin{center}   
    \LARGE
    \textbf{Dampak Positif dan Negatif Penerapan Teknologi IoT dalam Kehidupan Manusia}
  \end{center}
  \begin{center}
    \vspace{1cm}
    \includegraphics[scale=6]{logo_uajy}
    \vspace{0.5cm}
  \end{center}
  \begin{center}
    \Large
    Oleh:\\
    Natan Hari Pamungkas\\
    170709254
  \end{center}
\end{titlepage}

\justify
\setlength{\parindent}{4ex}
IoT atau \textit{Internet of Things} adalah teknologi yang menghubungkan objek sehari-hari ke dalam jaringan internet. Tidak hanya terhubung ke dalam jaringan internet, perangkat IoT juga harus mampu mengirimkan data, menerima data, dan dikendalikan lewat jaringan internet. Jika sebuah perangkat tidak dapat melakukan hal-hal tersebut, maka bisa disimpulkan bahwa perangkat tersebut bukanlah sebuah perangkat IoT.
\end{document}

