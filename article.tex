\documentclass[12pt, letterpaper]{article}
\usepackage[top=4cm, left=4cm, right=3cm, bottom=3cm]{geometry}
\usepackage{graphicx}
\usepackage[backend=biber, citestyle=ieee]{biblatex}
\usepackage{indentfirst}
\addbibresource{references.bib}
\linespread{1.25}
\graphicspath{ {./} }

\begin{document}
\begin{titlepage} 
  \begin{center}   
    \LARGE
    \textbf{Dampak Positif dan Negatif Penerapan Teknologi IoT dalam Kehidupan Manusia}
  \end{center}
  \begin{center}
    \vspace{1cm}
    \includegraphics[scale=6]{logo_uajy}
    \vspace{0.5cm}
  \end{center}
  \begin{center}
    \Large
    Oleh:\\
    Natan Hari Pamungkas\\
    170709254
  \end{center}
\end{titlepage}

\section*{Abstrak}
Pada zaman modern seperti sekarang ini, tidak ada aspek kehidupan manusia seperti sandang, pangan, dan papan yang tidak tersentuh oleh adanya teknologi. Hal ini termasuk juga teknologi IoT yang sekarang sudah mulai banyak digunakan tidak hanya oleh industri, tetapi juga oleh rumah tangga. Kehadiran perangkat-perangkat IoT seperti ini tentu saja dimaksudkan untuk memudahkan kehidupan manusia. Salah satu contoh dari perangkat IoT di dalam kehidupan sehari-hari adalah smart home atau rumah pintar berbasis IoT yang dapat dikontrol dan dimonitor melalui perangkat lunak smartphone.
\newline
\indent
Penerapan teknologi IoT memang dikenal luas dapat memberikan manfaat-manfaat yang positif bagi kehidupan umat manusia. Tetapi pada kenyataanya, segala sesuatu di dunia ini pasti selain memiliki sisi positif, juga memiliki sisi negatif, termasuk juga dalam hal ini teknologi IoT. Namun tentu saja dampak negatif yang dihasilkan sudah dilakukan pengkajian, sehingga dampak negatif tersebut dapat ditangani dan menjadi lebih sedikit daripada dampak positif yang dihasilkan oleh penerapan teknologi IoT ini.

\newpage
\section*{Pendahuluan}
IoT atau \textit{Internet of Things} adalah teknologi yang menghubungkan objek sehari-hari seperti televisi (TV), pemanggang roti, dan \textit{rice cooker} ke dalam jaringan internet.\cite{techtarget20} Tidak hanya terhubung ke dalam jaringan internet, perangkat IoT juga harus mampu mengirimkan data, menerima data, dan dikendalikan lewat jaringan internet. Jika sebuah perangkat tidak dapat melakukan hal-hal tersebut, maka bisa disimpulkan bahwa perangkat tersebut bukanlah sebuah perangkat IoT.
\newline
\indent
Sebuah perangkat IoT biasanya terdiri dari beberapa bagian seperti sensor, koneksi, pemrosesan data, dan antarmuka pengguna.\cite{dflair18} Sensor digunakan untuk mengambil data secara \textit{realtime}. Koneksi adalah media penghubung ke jaringan internet yang dapat berupa \textit{Wireles Fidelity} (WiFi), \textit{Bluetooth Low Energy} (BLE) atau bisa juga menggunakan jaringan \textit{Fifth Generation} (5G) yang kabarnya akan tersedia secara komersial pada tahun 2020.\cite{erricson17} Pemrosesan data berguna untuk memproses data yang
dikirimkan oleh sensor. Pemrosesan data dapat berupa sekedar mengkonversi data ataupun memproses data menggunakan algoritma pembelajaran mesin. Antarmuka pengguna berguna untuk merepresentasikan data yang telah diproses menjadi mudah untuk dipahami oleh pengguna. Pengguna disini tidak dibataskan hanya pada manusia, tetapi juga mesin, karena sebuah perangkat IoT juga dapat berkomunikasi secara \textit{machine to machine} (M2M).
\newline
\indent
Sejarahnya, IoT hanya merupakan sebatas konsep saja sampai pada tahun 1999.\cite{dversity16} Sebelumnya, pada awal tahun 1980-an, IoT sudah mulai diterapkan secara sederhana oleh para \textit{programmer} di Carnegie Mellon University pada mesin penjualan Coca-Cola. Sistem tersebut dapat memberi tahu apakah ada minuman yang tersedia atau tidak dan apakah minuman tersebut dingin atau tidak, sehingga mereka dapat memastikannya sebelum mengambil minuman ke mesin penjualan tersebut. Barulah pada tahun 1999, istilah IoT mulai digunakan
oleh Direktur Eksekutif Laboratorium Auto-ID Massachusetts Institute of Technology (MIT) yaitu Kevin Ashton saat sedang memberikan presentasi untuk perusahaan Procter \& Gamble (P\&G).

\newpage
\section{Dampak Positif IoT}
Dampak positif teknologi IoT yang diterapkan ke dalam kehidupam umat manusia sangat banyak jumlahnya. Dampak-dampak tersebut diataranya adalah:

\subsection{Memudahkan Manajemen Kota}
Konsep \textit{smart city} atau kota pintar sudah mulai digagas oleh banyak negara di dunia. Penerapan teknologi IoT yang dipadukan dengan \textit{cloud computing} atau komputasi awan ke dalam \textit{smart city} ini akan membarikan banyak manfaat baik untuk pemerintah maupun warga kota. \cite{icfit15}
\newline
\indent
Manfaat yang pertama adalah adanya \textit{Smart City Resource Management} (SCRM) yang efisien. Teknologi IoT berperan sangat penting dalam hal ini, karena dengan adanya IoT, data dari seluruh perangkat IoT seperti telepon genggam, mobil, bus kota, dan lain-lain dapat dikumpulkan dan diproses untuk membuat sistem seperti \textit{Traffic Mobility Management}. Sistem ini dapat membantu mobilitas warga kota menjadi lebih efisien. Data yang didapatkan dan diolah secara \textit{realtime}
dapat digunakan untuk membuat keputusan seperti rute-rute yang paling cepat untuk dilalui, tempat parkir yang tersedia dan bisa digunakan, dan juga dapat memberikan informasi tentang ketersediaannya transportasi umum.
\newline
\indent
Manfaat yang kedua adalah adanya \textit{Safety and Emergency Management} (SEM). Sistem ini memiliki \textit{City Safety and Accident Management} yang berguna untuk mengantisipasi maupun menangani kecelakaan. Contohnya adalah warga kota maupun pihak berwajin dapat mengakses informasi tentang daerah yang rawan kejahatan. Untuk mengantisipasi kecelakaan, peringatan juga akan dikirimkan ke pengguna kendaraan maupun pejalan kaki agar mereka waspada terhadap keberadaan kendaraan atau
pejalankaki sehingga tidak terjadi kecelakaan.
\newline
\indent
Selain untuk menangani kecelakaan, SEM juga memiliki \textit{Weather Risk and Alert Management}. Disini data yang didapatkan baik dari sensor-sensor bencana yang dipasang di kota maupun data yang dikirimkan dari gawai warga kota digunakan untuk membuat \textit{early warning system}. Dengan begitu, warga kota dapat meminimalisir dampak bencana karena telah memiliki kesiapan terlebih dahulu.
\newline
\indent
Manfaat yang terakhir adalah \textit{Citizen Health and Pleasant Enchancement} (CHPE). Sistem ini diharapkan dapat meningkatkan keaktifan warga kota dalam acara-acara yang diadakan di kota. Pemerintah kota dapat mengadakan acara yang sesuai dengan minat masyarakat melalui data yang dikirimkan melalui \textit{smartphone} masyarakat yang kemudian diolah. Masyarakat juga dapat mendaftarkan acaranya, misalnya ada yang membuat sebuah kompetisi, maka bisa juga dimasukan ke sistem sehingga
dapat menarik banyak peserta.
\newline
\subsection{Meningkatkan Produktivitas Pertanian}
Sektor pertaniang merupakan salah satu sektor penting untuk menjaga ketahanan pangan. Selama ini, sektor pertanian masih banyak dijalankan dengan sistem yang masih tergolong tradisional. Proses-proses seperti pendistribusian bibit, irigasi, dan pembasmian hama semuanya masih dilakukan secara konvensional. Hal tersebut membuat lambatnya produksi pangan yang dapat dihasilkan.
\newline
\indent
Untuk dapat meningkatkan produksi dari sektor pertanian, dapat diterapkan teknologi IoT. Penerapan teknologi IoT ini akan membuat sistem pertanian menjadi \textit{smart farming}. \textit{Smart farming} sendiri menggunakan sistem IoT yang saling terintegrasi, sehingga semua parameter seperti kelembaban tanah, hama, dan kebutuhan air semua dapat dimonitor dan dikontrol. Hal-hal seperti penyemprotan pestisida, pupuk, dan menjaga keamanan lahan pertanian juga dapat dilakukan secara otomatis
berdasarkan data-data dari sensor. Dengan begitu, produktivitas pertanian akan semakin meningkat berkat diterapkannya teknologi IoT.\cite{iemcon17} 
\newline
\subsection{\textit{E-Learning} berbasis IoT}
Pada zaman modern seperti sekarang, konsep \textit{e-learning} sudah menjadi hal yang lazim. Sudah banyak lembaga pendidikan baik yang formal maupun non formal menerapkan sistem pembelajaran tanpa tatap muka. Bahkan, ditengah-tengah pandemik COVID-19 ini, banyak institusi pendidikan menjalankan kegiatan belajar mengajarnya menggunakan konsep daring.
\newline
\indent
Namun, penerapan \textit{e-learning} sekarang masih memiliki banyak keterbatasan. Salah satunya untuk kegiatan belajar mengajar yang memerlukan alat-alat laboratorium akan sangat sulit dilakukan. Tidak semua peserta pembelajaran memiliki alat-alat yang dibutuhkan untuk melakukan kegiatan praktikum di rumah masing-masing. Hal ini tentu akan mengurangi pemahaman mereka dalam mempelajari suatu materi pembelajaran.
\newline
\indent
Dalam hal seperti ini, teknologi IoT dapat diterapkan pada konsep pembelajaran \textit{e-learning} ini. Teknologi IoT diterapkan sehingga objek-objek di laboratirium saling terkoneksi melalui jaringan internet dan dapat dikendalikan oleh para peserta pembelajaran. \cite{ijdr17} Data-data yang didapatkan dari objek-objek tersebut juga nantinya dapat digunakan oleh peserta pembelajaran untuk menyelesaikan tugas-tugas mereka. Dengan diterapkannya teknologi IoT, maka pembelajaran akan lebih
efisien dan lebih dapat dipahami oleh peserta pembelajaran.

\newpage
\section{Dampak Negatif}
Selain dampak positif, penerapan teknologi IoT ke dalam kehidupan manusia juga membawa dampak negatif yang diantarnya:

\subsection{Hilangnya Privasi}
Seperti yang sudah dibahas, sebuah perangkat IoT baru dapat dikatakan IoT jika perangkat tersebut terhubung ke jaringan internet dan dapat mengirimkan ataupun menerima data. Karena teknologi IoT melekat ke hampir seluruh objek sehari-hari, maka disini mulai timbul masalah privasi. 
\newline
\indent
Salah satu contohnya adalah teknologi jam anak berbasis IoT yang menggunakan teknologi \textit{Global Positioning System} (GPS). Pembuatan jam tangan ini dimaksudkan agar orang tua dapat mengetahui keberadaan anak-anak mereka. Teknologi seperti ini sebenarnya baik jika dibuat secara aman, karena orang tua dapat memantau posisi anak mereka secara \textit{real time} sehingga dapat mencegah hal-hal yang tidak diinginkan seperti penculikan ataupun kehilangan.
\newline
\indent
Dampak buruk yang ditimbulkan adalah ternyata komunikasi data yang terjadi pada jam tangan anak tersebut tidak dienkripsi. \cite{threatpost19} Karena posisi anak bisa dilacak secara \textit{real time}, maka tentunya ini dapat membuat anak tersebut dalam behaya. Pasalnya, orang-orang dengan niat jahat dapat dengan mudah mengetahui posisi anak yang memakai jam tersebut.

\subsection{Masalah Keamanan}
Masalah keamanan dalam teknologi memang tidak dapat dihindari. Sampai saat ini masih relatif banyak celah-celah keamanan yang ditemukan di produk-produk teknologi. Namun dalam hal teknologi IoT, masalah ini berbeda, karena lebih dari setengah dari perangkat IoT rentan terhadap serangan dengan kategori sedang sampai tinggi. \cite{threatpost20}
\newline
\indent
Menjaga keamanan perangkat IoT memang tergolong cukup sulit. Banyaknya sistem operasi yang digunakan, kebutuhan waktu \textit{up time} yang tinggi, sampai adanya kerentanan terhadap serangan lama menjadi tantangan dalam mengamankan perangkat-perangkat IoT. 
\newline
\indent
Kebanyakan sistem operasi yang dibuat untuk perangkat IoT tidak sesuai standard. Bahkan, kebanyakan sistem operasi tersebut tidak memiliki standard sama sekali, yang menjadi prioritas adalah fungsinya untuk terhubung ke internet. \cite{secledger17} Hal ini tentunya membuat banyak pengguna kwalahan dalam menjaga keamanan perangkat IoT mereka.
\newline
\indent
Mayoritas perangkat IoT juga memiliki kebutuhan \textit{up time} yang tinggi. Hal tersebut dikarenakan kebanyakan perangkat tersebut beroprasi pada \textit{mission critical position} yang mengharuskan perangkat-perangkat tersebut bekerja 24 jam selama tujuh hari. Hal ini menyulitkan untuk memperbarui sistem tersebut tanpa harus mematikannya. Masalah ini lah yang membuat beberapa perusahaan tidak memperbarui perangkat mereka selama bertahun-tahun.
\newline
\indent
Banyak juga perangkat IoT yang rentang terhadap metode serangan yang relatif sudah tua. Pada tahun 2018 terdapat \textit{malware} yang sudah berusia 10 tahun menginfeksi perangkat-perangkat IoT. \cite{futurum20} Target perangkat tersebut adalah alat-alat medis di rumah sakit yang berbasis IoT. Hal ini disebabkan karena pihak rumah sakit tidak memperbarui perangkat IoT mereka secara berkala.

\newpage
\section*{Kesimpulan}
Setiap penerapan teknologi pasti memiliki dampak yang positif maupun negatif. Hal ini berlaku juga untuk penerapan teknologi IoT. Dalam penerapannya, teknologi IoT ini memiliki begitu banyak manfaat seperti untuk manajemen kota, meningkatkan produktivitas di sektor pertanian, dan dalam hal pembelajaran elektronik. Sisi negatif dari diterapkannya teknologi IoT ini adalah adanya permasalahan di sektor privasi dan keamanan. Permasalahan ini sangat fatal, namun tetap bisa dihindari.
Misalnya dengan mengembangkan perangkat IoT yang dapat diperbarui tanpa mengganggu ketersediaan perangkat dan membuat sistem operasi standard yang selain memenuhi kebutuhan fungsional, juga memenuhi kriteria-kriteria keamanan.

\newpage
\printbibliography[title=Daftar Pustaka]
\end{document}

