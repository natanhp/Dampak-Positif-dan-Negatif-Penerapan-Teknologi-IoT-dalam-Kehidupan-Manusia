\documentclass[12pt, letterpaper]{article}
\usepackage[top=4cm, left=4cm, right=3cm, bottom=3cm]{geometry}
\usepackage{graphicx}
\usepackage[backend=biber, citestyle=ieee]{biblatex}
\usepackage{indentfirst}
\addbibresource{references.bib}
\linespread{1.25}
\graphicspath{ {./} }

\begin{document}
\begin{titlepage} 
  \begin{center}   
    \LARGE
    \textbf{Dampak Positif dan Negatif Penerapan Teknologi IoT dalam Kehidupan Manusia}
  \end{center}
  \begin{center}
    \vspace{1cm}
    \includegraphics[scale=6]{logo_uajy}
    \vspace{0.5cm}
  \end{center}
  \begin{center}
    \Large
    Oleh:\\
    Natan Hari Pamungkas\\
    170709254
  \end{center}
\end{titlepage}

\section*{Abstrak}
Pada zaman modern seperti sekarang ini, tidak ada aspek kehidupan manusia seperti sandang, pangan, dan papan yang tidak tersentuh oleh adanya teknologi. Hal ini termasuk juga teknologi IoT yang sekarang sudah mulai banyak digunakan tidak hanya oleh industri, tetapi juga oleh rumah tangga. Kehadiran perangkat-perangkat IoT seperti ini tentu saja dimaksudkan untuk memudahkan kehidupan manusia. Salah satu contoh dari perangkat IoT di dalam kehidupan sehari-hari adalah smart home atau rumah pintar berbasis IoT yang dapat dikontrol dan dimonitor melalui perangkat lunak smartphone.
\newline
\indent
Penerapan teknologi IoT memang dikenal luas dapat memberikan manfaat-manfaat yang positif bagi kehidupan umat manusia. Tetapi pada kenyataanya, segala sesuatu di dunia ini pasti selain memiliki sisi positif, juga memiliki sisi negatif, termasuk juga dalam hal ini teknologi IoT. Namun tentu saja dampak negatif yang dihasilkan sudah dilakukan pengkajian, sehingga dampak negatif tersebut dapat ditangani dan menjadi lebih sedikit daripada dampak positif yang dihasilkan oleh penerapan teknologi IoT ini.

\newpage
\section*{Pendahuluan}
IoT atau \textit{Internet of Things} adalah teknologi yang menghubungkan objek sehari-hari seperti televisi (TV), pemanggang roti, dan \textit{rice cooker} ke dalam jaringan internet.\cite{techtarget20} Tidak hanya terhubung ke dalam jaringan internet, perangkat IoT juga harus mampu mengirimkan data, menerima data, dan dikendalikan lewat jaringan internet. Jika sebuah perangkat tidak dapat melakukan hal-hal tersebut, maka bisa disimpulkan bahwa perangkat tersebut bukanlah sebuah perangkat IoT.
\newline
\indent
Sebuah perangkat IoT biasanya terdiri dari beberapa bagian seperti sensor, koneksi, pemrosesan data, dan antarmuka pengguna.\cite{dflair18} Sensor digunakan untuk mengambil data secara \textit{realtime}. Koneksi adalah media penghubung ke jaringan internet yang dapat berupa \textit{Wireles Fidelity} (WiFi), \textit{Bluetooth Low Energy} (BLE) atau bisa juga menggunakan jaringan \textit{Fifth Generation} (5G) yang kabarnya akan tersedia secara komersial pada tahun 2020.\cite{erricson17} Pemrosesan data berguna untuk memproses data yang
dikirimkan oleh sensor. Pemrosesan data dapat berupa sekedar mengkonversi data ataupun memproses data menggunakan algoritma pembelajaran mesin. Antarmuka pengguna berguna untuk merepresentasikan data yang telah diproses menjadi mudah untuk dipahami oleh pengguna. Pengguna disini tidak dibataskan hanya pada manusia, tetapi juga mesin, karena sebuah perangkat IoT juga dapat berkomunikasi secara \textit{machine to machine} (M2M).
\newline
\indent
Sejarahnya, IoT hanya merupakan sebatas konsep saja sampai pada tahun 1999.\cite{dversity16} Sebelumnya, pada awal tahun 1980-an, IoT sudah mulai diterapkan secara sederhana oleh para \textit{programmer} di Carnegie Mellon University pada mesin penjualan Coca-Cola. Sistem tersebut dapat memberi tahu apakah ada minuman yang tersedia atau tidak dan apakah minuman tersebut dingin atau tidak, sehingga mereka dapat memastikannya sebelum mengambil minuman ke mesin penjualan tersebut. Barulah pada tahun 1999, istilah IoT mulai digunakan
oleh Direktur Eksekutif Laboratorium Auto-ID Massachusetts Institute of Technology (MIT) yaitu Kevin Ashton saat sedang memberikan presentasi untuk perusahaan Procter \& Gamble (P\&G).

\newpage
\printbibliography[title=Daftar Pustaka]
\end{document}

